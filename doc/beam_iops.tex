
\documentclass{article}
\usepackage{fontspec}
\usepackage{multicol}
\usepackage{textcomp}

\title{BEAM instructions semantics}

\setromanfont[Mapping=tex-text]{NeoSans}

\newcommand{\tmpa}{\textbf{tmp\_arg1}}
\newcommand{\tmpb}{\textbf{tmp\_arg2}}
\newcommand{\ip}{\textbf{IP}}
\newcommand{\cp}{\textbf{CP}}
\newcommand{\ep}{\textbf{EP}}
\newcommand{\htop}{\textbf{HTOP}}
\newcommand{\save}{\textbf{SAVE}}
\newcommand{\iop}[1]{\texttt{#1}}

\begin{document}
\maketitle

\section*{Abstract machine elements}

\begin{enumerate}
\item Registers. R0 is a special fast register.
\item \ip{} (instruction pointer) contains the reference to the instruction
being executed.
\item \cp{} (continuation pointer) contains a return address of the current
function call.
\item The stack$:$

\setlength{\unitlength}{8pt}
\begin{picture}(20,20)
\put(5,0){\line(0,1){20}}
\put(15,0){\line(0,1){20}}
\multiput(5,0)(0,2){11}{\line(1,0){10}}
\put(0,4.6){\ep{}}
\put(2,5){\vector(1,0){3}}
\multiput(8,0.6)(0,2){2}{unused}
\put(8,4.6){saved \cp{}}
\put(8,6.6){slot 0}
\put(8,8.6){slot 1}
\put(8,10.6){slot 2}
\end{picture}

\item \ep{} (stack pointer) points to the last value pushed on stack.
\item \htop{} (heap top) points to the first empty word of the heap.
\item The message queue$:$

\begin{picture}(20,16)
\put(5,0){\line(0,1){16}}
\put(15,0){\line(0,1){16}}
\multiput(5,0)(0,2){9}{\line(1,0){10}}
\put(0,6.6){\save{}}
\put(4,7){\vector(1,0){1}}
\end{picture}

\item \save{} pointer marks the message being currently matched.
\item \tmpa{} and \tmpb{} variables that temporarily hold values for the
imminent arithmetic or logical operation.
\end{enumerate}

\section*{Instructions semantics}

\subsection*{\iop{allocate t t}}

Allocates Arg0 slots on the stack. Arg1 is Live. See Fig. \ref{fig:alloc}.

\begin{figure}[!ht]
\setlength{\unitlength}{8pt}
\begin{picture}(40,20)
\put(5,0){\line(0,1){20}}
\put(15,0){\line(0,1){20}}
\multiput(5,0)(0,2){11}{\line(1,0){10}}
\put(0,12.6){\ep{}}
\put(2,13){\vector(1,0){3}}
\multiput(8,0.6)(0,2){6}{unused}
\put(25,0){\line(0,1){20}}
\put(35,0){\line(0,1){20}}
\multiput(25,0)(0,2){11}{\line(1,0){10}}
\put(20,4.6){\ep{}}
\put(22,5){\vector(1,0){3}}
\multiput(28,0.6)(0,2){2}{unused}
\put(28,4.6){saved \cp{}}
\multiput(28,6.6)(0,2){3}{allocated}
\put(18,10){\vector(1,0){4}}
\end{picture}
\caption{Allocating 3 slots on the stack.}
\label{fig:alloc}
\end{figure}

\subsection*{\iop{allocate\_heap t I t}}

Allocates Arg0 slots on the stack. See \iop{allocate t t}. Ensures that heap
contains at least Arg1 words after \htop{}. Arg2 is Live.

\subsection*{\iop{deallocate I}}

Restores \cp{} to the value references by \ep{}. Adds Arg0$+$1 to \ep{} to
remove the saved \cp{} and Arg0 slots from the stack.

\subsection*{\iop{init y}}

Sets the Arg0 slot to nil. The op was called ``kill'' earlier.

\subsection*{\iop{init2 y y}}

Sets slots Arg0 and Arg1 to nil.

\subsection*{\iop{init3 y y y}}

Sets slots Arg0, Arg1, and Arg2 to nil.

\subsection*{\iop{i\_trim I}}

Removes Arg0 slots from the stack. Th saved \cp{} is moved to the new top of the
stack.

\subsection*{\iop{test\_heap I t}}

Ensures that the heap has at least Arg0 words available. Arg1 is Live.

\subsection*{\iop{allocate\_zero t t}}

Same as \iop{allocate t t} but sets the slots to nil.

\subsection*{\iop{allocate\_heap\_zero t I t}}

Same as \iop{allocate\_heap t I t} but sets the slots to nil.

\subsection*{\iop{i\_select\_val r f I}\\
\iop{i\_select\_val x f I}\\
\iop{i\_select\_val y f I}}

Searches an array of \verb${val,addr}$ pairs for the value of Arg0. The length
of the array is Arg2. If the value is found the execution continues at the
corresponding \verb$addr$. Otherwise, jumps to Arg1. Uses binary search and
\emph{bitwise} comparisons.

\subsection*{\iop{i\_select\_val2 r f c f c f}\\
\iop{i\_select\_val2 x f c f c f}\\
\iop{i\_select\_val2 y f c f c f}}

Same as \iop{i\_select\_val rxy f I} but uses an array made of
\verb${Arg2,Arg3}$ and \verb${Arg4,Arg5}$.

\subsection*{\iop{i\_jump\_on\_val rxy f I I}}

Implements a ``jump table''. Calculates the index by adding Arg0 to the minimum
value Arg2. Jumps to Arg1 if the index falls outside of the table boundaries.
Arg3 is the length of the jump table. Picks the label from the table using the
calculated index and jumps there. 

\subsection*{\iop{i\_jump\_on\_val\_zero rxy f I}}

Same as \iop{i\_jump\_on\_val rxy f I I} but the minimum value is assumed to be
zero.

\subsection*{\iop{i\_select\_tuple\_arity r f I}\\
\iop{i\_select\_tuple\_arity x f I}\\
\iop{i\_select\_tuple\_arity y f I}}

Checks that Arg0 is tuple and then searches the array for \verb${arity,addr}$
for the right arity. See \iop{i\_select\_val rxy f I}.

\subsection*{\iop{i\_select\_tuple\_arity2 r f A f A f}\\
\iop{i\_select\_tuple\_arity2 x f A f A f}\\
\iop{i\_select\_tuple\_arity2 y f A f A f}}

Same as \iop{i\_select\_tuple\_arity rxy f I} but uses the array made up of two
pairs$:$ \verb${Arg2,Arg3}$ and \verb${Arg4,Arg5}$.

\subsection*{\iop{i\_func\_info I a a I}}

Raises \texttt{function\_clause} exception. Arg1, Arg2, Arg3 are MFA. Arg0 is
a mystery coming out of nowhere, probably, needed for NIFs.

\subsection*{\iop{return}}

\ip{} is set to \cp{}.

\subsection*{\iop{get\_list rxy rxy rxy}}

Gets the head of the list from Arg0 and puts it to Arg1, the tail --- to Arg2.

\subsection*{\iop{catch y f}}

Creates a new ``catch'' context. Saves value of the label from Arg1 to the Arg0 stack
slot.

\subsection*{\iop{catch\_end y}}

Pops a ``catch'' context. Erases the label saved in the Arg0 slot. Noval in R0
indicates that something is caught. If R1 contains atom \verb$throw$ then R0 is
set to R2. If R1 contains atom \verb$error$ R0 is set to
\verb${'EXIT',{R2,<stack trace>}}$.

\subsection*{\iop{try\_end y}}

Pops a ``catch'' context. Erases the label saved in the Arg0 slot. Noval in R0
indicate that something is caught. If so, R0 is set to R1, R1 --- to R2, R2 ---
to R3.

\subsection*{\iop{try\_case\_end s}}

Raises a \verb$try_clause$ exception with the value read from Arg0.

\subsection*{\iop{set\_tuple\_element s d P}}

Sets the Arg2'th element of the tuple Arg1 to to the value of Arg0. The index in
Arg2 is 1-based.

\subsection*{\iop{i\_get\_tuple\_element rxy P rxy}}

Gets the value of the the Arg1'th element of the tuple Arg0 and puts it to Arg2.
Arg1 is 1-based.

\subsection*{\iop{is\_number f rxy}}

Jumps to Arg0 if Arg1 is not integer and is not float.

\subsection*{\iop{jump f}}

Continues execution at the location Arg0.

\subsection*{\iop{case\_end rxy}}

Raises the \verb$case_clause$ exception with the value of Arg0.

\subsection*{\iop{badmatch rxy}}

Raises the \verb$badmatch$ exception with the value of Arg0.

\subsection*{\iop{if\_end}}

Raises the \verb$if_clause$ exception.

\subsection*{\iop{raise s s}}

Raises the exception. The instruction is garbled by backward compatibility. Arg0
is a stack trace and Arg1 is the value accompanying the exception. The reason of
the raised exception is dug up from the stack trace. 

\subsection*{\iop{badarg j}}

If Arg0 is not 0 then jumps to the label Arg0. Otherwise, raises the
\verb$badarg$ exception.

\subsection*{\iop{system\_limit j}}

Similar to \iop{badarg j} but raises \verb${system_limit}$ exception.

\subsection*{\iop{move\_jump f ncxy}}

Move the value from Arg1 to R0 and then jumps to the label Arg0.

\subsection*{\iop{move\_x1 c}}

Sets R1 to the value of Arg0.

\subsection*{\iop{move\_x2 c}}

Sets R2 to the value of Arg0.

\subsection*{\iop{move2 x y x y}\\
\iop{move2 y x y x}\\
\iop{move2 x x x x}}

Combine two move operations.

\subsection*{\iop{move rxync rxy}}

Copies Arg0 to Arg1.

\subsection*{\iop{recv\_mark f}}

Saves the last pointer of the message queue and the label of the soon expected
\iop{loop\_rec f r}.

\subsection*{\iop{i\_recv\_set f}}

Sets the \save{} pointer to the saved last pointer if the label is right. This
is somehow needed for in-order processing of monitoring messages.

\subsection*{\iop{remove\_message}}

Removes the (matched) message from the message queue.

\subsection*{\iop{timeout}}

Timeout has occured. Resets the \save{} pointer to the first message in the
message queue.

\subsection*{\iop{timeout\_locked}}

Same as \iop{timeout} but with SMP precautions.

\subsection*{\iop{i\_loop\_rec f r}}

Picks the next message from the message queue and places it in R0. If there are
no messages then jumps to Arg0 which points to \iop{wait} or \iop{wait\_timeout}
instruction.

\subsection*{\iop{loop\_rec\_end f}}

Advances the \save{} pointer to the next message as the current message did not
match. Jumps to Arg0 that points to \iop{loop\_rec f r} instruction.

\subsection*{\iop{wait f}}

Prepares to wait indefinitely for a message to arrive. When the message arrives
transfers control to the \iop{loop\_rec} instruction at the label Arg0.

\subsection*{\iop{wait\_locked f}}

Same as \iop{wait f}.

\subsection*{\iop{wait\_unlocked f}}

Same as \iop{wait f} but messes with SMP locks.

\subsection*{\iop{i\_wait\_timeout f I}\\
\iop{i\_wait\_timeout f s}}

Sets the timer to the appropriate time as indicated by Arg1. Afterwards acts as
\iop{wait f}.

\subsection*{\iop{i\_wait\_timeout\_locked f I}\\
\iop{i\_wait\_timeout\_locked f s}}

Same as \iop{i\_wait\_timeout f I} but with SMP trickery.

\subsection*{\iop{i\_wait\_error}}

Raises \verb$timeout_value$ exception.

\subsection*{\iop{i\_wait\_error\_locked}}

Same as \iop{i\_wait\_error} but with due respect to SMP.

\subsection*{\iop{send}}

Sends the message from R1 to the process R0.

\subsection*{\iop{i\_is\_eq\_exact\_immed f rxy c}}

Checks if Arg1 equals Arg2 using \emph{bitwise} comparison. Jumps to Arg0 it
does not.

\subsection*{\iop{i\_is\_ne\_exact\_immed f rxy c}}

Checks if Arg1 equals Arg2 using \emph{bitwise} comparison. Jumps to Arg0 it
does.

\subsection*{\iop{i\_is\_eq\_exact\_literal f rxy c}}

Checks if Arg1 equals Arg2 using the term comparison. Jumps to Arg0 it does not.

\subsection*{\iop{i\_is\_ne\_exact\_literal f rxy c}}

Checks if Arg1 equals Arg2 using the term comparison. Jumps to Arg0 it does.

\subsection*{\iop{i\_is\_eq\_exact f}}

Checks if \tmpa{} ``exactly'' equals \tmpb{}. Jumps to the label Arg0 if it does
not.

\subsection*{\iop{i\_is\_ne\_exact f}}

Checks if \tmpa{} ``exactly'' equals \tmpb{}. Jumps to the label Arg0 it does.

\subsection*{\iop{i\_is\_lt f}}

Checks if \tmpa{} is less than \tmpb{}. Jumps to the label Arg0 if it is not.

\subsection*{\iop{i\_is\_ge f}}

Checks if \tmpa{} is greater than or equal to \tmpb{}. Jumps to the label Arg0
if it is not.

\subsection*{\iop{i\_is\_eq f}}

Checks if \tmpa{} is equal to \tmpb{}. Jumps to the label Arg0 if it is not.

\subsection*{\iop{i\_is\_ne f}}

Checks if \tmpa{} is equal to \tmpb{}. Jumps to the label Arg0 if it is.

\subsection*{\iop{i\_put\_tuple rxy I}}

Creates a tuple on the heap and places it in Arg0. Arg1 is the arity of the
tuple. The elements of the tuple follow the instruction. Some of the elements
may refer to registers or stack slots.

\subsection*{\iop{put\_list s s d}}

Create a cons cell on the heap and places in onto Arg2. Arg0 is the head of the
list saved to \htop{}[0] and Arg1 is the tail saved to \htop{}[1].

\subsection*{\iop{i\_fetch s s}}

Fetches values from Arg0 and Arg1 and places them in \tmpa{} and \tmpb{}
respectively.

\subsection*{\iop{move\_return xcn r}}

Combines \iop{move xcn r} and \iop{return}.

\subsection*{\iop{move\_deallocate\_return xycn r Q}}

Combines \iop{move xycn r}, \iop{deallocate Q}, and \iop{return}.

\subsection*{\iop{deallocate\_return Q}}

Combines \iop{deallocate Q} and \iop{return}.

\subsection*{\iop{test\_heap\_1\_put\_list I y}}

Test that the heap has a room for Arg0 words and than conses value of Arg1 with
R0.

\subsection*{\iop{is\_tuple\_of\_arity f rxy A}}

Jumps to the label Arg0 if Arg1 is not a tuple or its arity is not Arg2. If Arg1
is tuple its value is saved to \tmpa{}.

\subsection*{\iop{is\_tuple f rxy}}

Jumps to the label Arg0 if Arg1 is not a tuple.

\subsection*{\iop{test\_arity f rxy A}}

Jumps to the label Arg0 is the arity of the tuple Arg1 is not Arg2. The value of
Arg1 is saved to \tmpa{}.

\subsection*{\iop{extract\_next\_element xy}}

The pointer in \tmpa{} moved to the next element of the tuple. The element is
read from the location pointed to by \tmpa{} to Arg0.

\subsection*{\iop{extract\_next\_element2 xy}}

The pointer in \tmpa{} moved to the next element of the tuple. Copies 2 terms
from the array pointed to by \tmpa{} to the registers or slots pointed to by
Arg0. For instance, if Arg0 is R3 then tuple elements are loaded to R3 and R4.

\subsection*{\iop{extract\_next\_element3 xy}}

Same as \iop{extract\_next\_element2 xy} but copies three elements.

\subsection*{\iop{is\_integer\_allocate f rx I I}}

Combines \iop{is\_integer f rx} and \iop{allocate I I}.

\subsection*{\iop{is\_integer f rxy}}

Jumps to the label Arg0 if Arg1 is not integer.

\subsection*{\iop{is\_list f rxy}}

Jumps to the label Arg0 if Arg1 is not a list.

\subsection*{\iop{is\_nonempty\_list\_allocate f rx I t}}

Combines \iop{is\_nonempty\_list f rx} and \iop{allocate I t}.

\subsection*{\iop{is\_nonempty\_list\_test\_heap f r I t}}

Combines \iop{is\_nonempty\_list f r} and \iop{test\_heap I t}.

\subsection*{\iop{is\_nonempty\_list f rxy}}

Jumps to the label Arg0 if Arg1 is not an nonempty list.

\subsection*{\iop{is\_atom f rxy}}

Jumps to the label Arg0 if Arg1 is not an atom.

\subsection*{\iop{is\_float f rxy}}

Jumps to the label Arg0 if Arg1 is not a float.

\subsection*{\iop{is\_nil f rxy}}

Jumps to the label Arg0 if Arg1 is not nil.

\subsection*{\iop{is\_bitstring f rxy}}

Jumps to the label Arg0 if Arg1 is not a bitstring (or a binary).

\subsection*{\iop{is\_reference f rxy}}

Jumps to the label Arg0 if Arg1 is not a reference.

\subsection*{\iop{is\_pid f rxy}}

Jumps to the label Arg0 if Arg1 is not a pid.

\subsection*{\iop{is\_port f rxy}}

Jumps to the label Arg0 if Arg1 is not a port.

\subsection*{\iop{is\_boolean f rxy}}

Jumps to the label Arg0 if Arg1 is not true or false.

\subsection*{\iop{is\_function2 f s s}}

Jumps to the label Arg0 if Arg1 is not fun or have an arity different from Arg1.

\subsection*{\iop{allocate\_init t I y}}

Combines \iop{allocate t I} and \iop{init y}.

\subsection*{\iop{i\_apply}}

Applies the function identified by R0 (module), R1 (function), and R2
(arguments). Sets \cp{} to the point after the instruction.

\subsection*{\iop{i\_apply\_last P}}

Applies the function identified by R0 (module), R1 (function), and R2
(arguments). Afterwards, restores \cp{} from stack and deallocates Arg0 slots
from stack.

\subsection*{\iop{i\_apply\_only}}

Applies the function identified by R0 (module), R1 (function), and R2
(arguments).

\subsection*{\iop{apply I}}

Finds the address from the export entry for the function (possibly BIF)
identified by the module R(N), the function R($N+1$), and the arity N. The arity
is taken from Arg0. Sets \cp{} to the next instruction and jumps to the found
address.

\subsection*{\iop{apply\_last I P}}

Similar to \iop{apply I} but restores \cp{} from stack and drops Arg1 slots from
it before jumping to the found address.

\subsection*{\iop{i\_apply\_fun}}

Applies the fun R0 to the argument list in R1. Set \cp{} to the location right
after the instruction.

\subsection*{\iop{i\_apply\_fun\_last P}}

Applies the fun R0 to the argument list in R1. Restores \cp{} from stack and
deallocates Arg0 slots from stack.

\subsection*{\iop{i\_apply\_fun\_only}}

Applies the fun R0 to the argument list in R1.

\subsection*{\iop{i\_hibernate}}

Puts the process into a hibernated state dropping its stack and minimizing its
heap. When a message arrives or if the message queue is not empty the process is
waken up and starts executing a function identified by R0 (module), R1
(function), and R2 (arguments).

\subsection*{\iop{call\_bif0 e}}

Calls the BIF from Arg0 without arguments and returns result in R0.

\subsection*{\iop{call\_bif1 e}}

Calls the BIF from Arg0 with R0 as the argument and returns result in R0.

\subsection*{\iop{call\_bif2 e}}

Calls the BIF from Arg0 with arguments in R0 and R1 and returns result in R0.

\subsection*{\iop{call\_bif3 e}}

Calls the BIF from Arg0 with arguments in R0, R1, and R2 and returns result in R0.

\subsection*{\iop{i\_get s d}}

Reads the process dictionary key Arg0 and places the result to Arg1.

\subsection*{\iop{self rxy}}

Put the identifier of the current process to Arg0.

\subsection*{\iop{node rxy}}

Put the node identifier of the virtual machine to Arg0.

\subsection*{\iop{i\_fast\_element rxy j I d}}

Gets the Arg2'th element of the tuple Arg0 and places it to Arg3. Arg1 is
effectively disregarded.

\subsection*{\iop{i\_element rxy j s d}}

Gets the Arg2'th element of the tuple Arg0 and places it to Arg3. Arg1 is
effectively disregarded.

\subsection*{\iop{bif1 f b s d}}

Calls the BIF from Arg1 with the argument Arg2 and places the result to Arg3.
Upon error jumps to Arg0 as appropriate for the guard BIF.

\subsection*{\iop{bif1\_body b s d}}

Calls the BIF from Arg0 with the argument Argr1 and places the result to Arg3.
This is the case of a guard BIF in the function body. It fails like an ordinary
BIF.

\subsection*{\iop{i\_bif2 f b d}}

Calls the guard BIF from Arg1 with the two arguments in \tmpa{} and \tmpb{} and
places the result to Arg2. Upon error jumps to Arg0 as appropriate for the guard
BIF.

\subsection*{\iop{i\_bif2\_body b d}}

Calls the guard BIF from Arg0 with the two arguments in \tmpa{} and \tmpb{} and
places the result to Arg1. This is the case of a guard BIF in the function body.
It fails like an ordinary BIF.

\subsection*{\iop{i\_move\_call c r f}}

Combines \iop{move c r} and \iop{call f}.

\subsection*{\iop{i\_move\_call\_last f P c r}}

Combines \iop{move c r} and \iop{call\_last f P}.

\subsection*{\iop{i\_move\_call\_only f c r}}

Combines \iop{move c r} and \iop{call\_only f}.

\subsection*{\iop{move\_call xy r f}}

Same as \iop{i\_move\_call xy r f}.

\subsection*{\iop{move\_call\_last xy r f Q}}

Same as \iop{i\_move\_last f Q xy r}.

\subsection*{\iop{move\_call\_only x r f}}

Same as \iop{i\_move\_only xy r f}.

\subsection*{\iop{i\_call f}}

Sets \cp{} to the location after the instruction and then jumps to the label
Arg0.

\subsection*{\iop{i\_call\_last f P}}

Restores \cp{} from the stack, drops Arg1 slots from stack, and jumps to the
label Arg0.

\subsection*{\iop{i\_call\_only f}}

Jumps to Arg0.

\subsection*{\iop{i\_call\_ext e}}

Sets \cp{} to the point after the instruction and then jumps to the address of
the export entry Arg0.

\subsection*{\iop{i\_call\_ext\_last e P}}

Restores \cp{} from the stack, drops Arg1 slots from the stack, and then jumps
to the address of the export entry Arg0.

\subsection*{\iop{i\_call\_ext\_only e}}

Jumps to the address of the export entry Arg0.

\subsection*{\iop{i\_move\_call\_ext c r e}}

Combines \iop{move c r} with \iop{i\_call\_ext e}.

\subsection*{\iop{i\_move\_call\_ext\_last e P c r}}

Combines \iop{move c r} with \iop{i\_call\_last e}.

\subsection*{\iop{i\_move\_call\_ext\_only e c r}}

Combines \iop{move c r} with \iop{i\_call\_only e}.

\subsection*{\iop{i\_call\_fun I}}

Calls the fun with the arity Arg0. The fun is found in the arity'th registers.
Thus the fun call arguments are in registers starting with R0 and the fun itself
immediately follows. The result is returned in R0. Sets \cp{} to the point after
the instruction and jumps to the fun's entry.

\subsection*{\iop{i\_call\_fun\_last I P}}

Same as \iop{i\_call\_fun I} but restores \cp{} from stack and drops Arg1 slots
from it before proceeding to the fun's entry.

\subsection*{\iop{i\_make\_fun I t}}

Creates a fun and puts it to R0. Arg0 contains a reference to the entry of the
lamdba table of the module. Arg1 is the number of free variables. The values of
free variables are taken from registers starting with R0. Only these registers
are concidered live. It means that Arg1 is Live also.

\subsection*{\iop{is\_function f rxy}}

Jumps to the label Arg0 if Arg1 is not a fun.

\subsection*{\iop{i\_bs\_start\_match2 rxy f I I d}}

Checks that Arg0 is the matching context with enough slots for saved offsets.
The number of slots needed is given by Arg3. If there not enough slots of if
Arg0 is a regular binary then recreates the matching context and saves it to
Arg4. If something does not work jumps to Arg1. Arg2 is Live.

\subsection*{\iop{i\_bs\_save2 rx I}}

Saves the offset from the matching context Arg0 to the saved offsets array at
the index Arg1.

\subsection*{\iop{i\_bs\_restore2 rx I}}

Does the opposite of \iop{i\_bs\_save2 rx I}. Restores the offset of the
matching context Arg0 from the saved offsets array at index Arg1.

\subsection*{\iop{i\_bs\_match\_string rx f I I}}

Compares Arg2 bits from the matching context Arg2 with the data referenced by
the raw data pointer Arg3. Jumps to the label Arg1 if there is no match.

\subsection*{\iop{i\_bs\_get\_integer\_small\_imm rx I f I d}}

Gets a (small) integer of size (in bits) Arg1 from the matching context Arg0 and
puts it to Arg4. Arg3 are flags. Jumps to the label Arg2 on failure.

\subsection*{\iop{i\_bs\_get\_integer\_imm rx I I f I d}}

Gets an integer of size (in bits) Arg1 from the matching context Arg0 and puts
it to Arg5. Arg4 are flags. Arg2 is Live. Jumps to the label Arg3 on failure.

\subsection*{\iop{i\_bs\_get\_integer f I I d}}

Gets an integer of size \tmpb{} (in units) from the matching context \tmpa{} and
saves it to Arg3. Arg1 is Live.  Arg2 are flags and unit. Jumps to Arg0 on
failure.

\subsection*{\iop{i\_bs\_get\_integer\_8 rx f d}}

Gets an 8-bit integer from the matching context Arg0 and saves it to Arg2. Jumps
to Arg1 if there are not enough bits left.

\subsection*{\iop{i\_bs\_get\_integer\_16 rx f d}}

Gets an 16-bit integer from the matching context Arg0 and saves it to Arg2. Jumps
to Arg1 if there are not enough bits left.

\subsection*{\iop{i\_bs\_get\_integer\_32 rx f I d}}

Gets an 32-bit integer from the matching context Arg0 and saves it to Arg3. Jumps
to Arg1 if there are not enough bits left. Arg2 is Live as the integer may not
be small.

\subsection*{\iop{i\_bs\_get\_binary\_imm2 f rx I I I d}}

Gets a binary from the matching context Arg1 of size (in bits) Arg3. Arg2 is
Live. Arg4 are flags. The binary is put to Arg5. Jumps yo Arg0 on failure.

\subsection*{\iop{i\_bs\_get\_binary2 f rx I s I d}}

Gets a binary from the matching context Arg1 of size (in units) Arg3. Arg2 is
Live. Arg4 are flags and unit. Puts result to Arg5. Jumps to Arg0 on failure.

\subsection*{\iop{i\_bs\_get\_binary\_all2 f rx I I d}}

Gets the rest of the matching context Arg1 as a binary and puts it to Arg4. The
binary size should be divisable by unit Arg3. Arg2 is Live. Jumps to Arg0 upon
failure.

\subsection*{\iop{i\_bs\_get\_binary\_all\_reuse rx f I}}

Replaces the matching context Arg0 with the rest of its reminder represented as
a binary. Arg2 is the unit. The size of the binary should be divisible by unit.
Jumps to the label Arg1 on failure.

\subsection*{\iop{i\_bs\_get\_float2 f rx I s I d}}

Gets the float value form the matching context Arg1. Arg3 is the size in units.
Arg4 are unit and flags. Puts the value to Arg5. Arg2 is Live. Jumps to Arg0 on
failure.

\subsection*{\iop{i\_bs\_skip\_bits2\_imm2 f rx I}}

Move the offset of the matching context Arg1 by Arg2 bits. Jumps to Arg0 on
failure.

\subsection*{\iop{i\_bs\_skip\_bits2 f rx rxy I}}

Skips2 Arg2 units of the matching context Arg1. Arg3 is the unit size. Jumps to
Arg0 on failure.

\subsection*{\iop{i\_bs\_skip\_bits\_all2 f rx I}}

Skips all remaining bits of the matching context Arg1. The number of bits
skipped should be divisible by unit Arg2. Jumps to the label Arg0 on failure.

\subsection*{\iop{bs\_test\_zero\_tail2 f rx}}

Jumps to the label in Arg0 if the matching context Arg1 still have unmatched
bits.

\subsection*{\iop{bs\_test\_tail\_imm2 f rx I}}

Checks that the matching context Arg1 has exactly Arg2 unmatched bits. Jumps to
the label Arg0 if it is not so.

\subsection*{\iop{bs\_test\_unit f rx I}}

Checks that the size of the remainder of the matching context Arg1 is divisible
by unit Arg2. Jumps to the label Arg0 if it is not so.

\subsection*{\iop{bs\_test\_unit8 f rx}}

Checks that the size of the remainder of the matching context Arg1 is divisible
by 8. Jumps to the label Arg1 if it is not.

\subsection*{\iop{bs\_context\_to\_binary rxy}}

Converts the matching context to a (sub)binary using almost the same code as
\iop{i\_bs\_get\_binary\_all\_reuse rx f I}.

\subsection*{\iop{i\_bs\_get\_utf8 rx f d}}

Extracts a UTF-8 encoded character from the matching context Arg0 and places it
to Arg2. Jumps to Arg1 upon failure.

\subsection*{\iop{i\_bs\_get\_utf16 rx f I d}}

Extracts a UTF-16 encoded character from the matching context Arg0 using flags
Arg2 and places it to Arg3. Jumps to Arg1 upon failure.

\subsection*{\iop{i\_bs\_validate\_unicode\_retract j}}

Verifies that a matched out integer \tmpa{} falls into the valid range for
Unicode characters. If not the matching context \tmpb{} is rectracted by 32
bits. Arg0 looks like a failure label but it is never used.

\subsection*{\iop{i\_bs\_init\_fail rxy j I d}}

The same as \iop{i\_bs\_init I I d} but may fail if the size in Arg0 is invalid.
Arg1 is not used.

\subsection*{\iop{i\_bs\_init\_fail\_heap I j I d}}

Follows the \iop{i\_fetch d d} that loads binary data size into \tmpa{}.
Allocates space for a shared binary data of \tmpa{} size. Creates a ProcBin
referencing the data. Makes sure that heap has enough space for a subbinary and
Arg0 more words. Arg2 is Live. Puts the ProcBin to Arg3.

\subsection*{\iop{i\_bs\_init I I d}}

Allocates space for a shared binary data of Arg0 size. Creates a ProcBin
referencing the data. Makes sure that the heap has enough space for a subbinary.
Arg1 is Live. Puts the ProcBin into Arg2.

\subsection*{\iop{i\_bs\_init\_heap I I I d}}

Allocates space for a shared binary data of Arg0 size. Create a ProcBin
referencing the data. Makes sure that the heap has enough space for a subbinary
and Arg1 more words. Arg2 is Live. Puts the ProcBin into Arg3.

\subsection*{\iop{i\_bs\_init\_heap\_bin I I d}}

Allocates a heap binary of size Arg0 and makes sure that the heap has enough
space for a subbinary. Arg1 is Live. The heap binary is put to Arg2.

\subsection*{\iop{i\_bs\_init\_heap\_bin\_heap I I I d}}

Allocates a heap binary of size Arg0 and makes sure that the heap has enough
space for a subbinary and Arg1 more words. Arg2 is Live. The heap binary is put
to Arg3.

\subsection*{\iop{i\_bs\_init\_bits\_fail rxy j I d}}

Same as \iop{i\_bs\_init\_bits\_heap I I I d} but the number of extra words
allocated on the heap is zero and may fail if the number of bits in Arg0 is not
valid. Arg1 is not used.

\subsection*{\iop{i\_bs\_init\_bits\_fail\_heap I j I d}}

Same as \iop{i\_bs\_init\_bits\_heap I I I d} but the number of bits is fetched
to \tmpa{} by previous instruction and may fail if the number of bits in \tmpa{}
is not valid. Arg1 is not used.

\subsection*{\iop{i\_bs\_init\_bits I I d}}

Same as \iop{i\_bs\_init\_bits\_heap I I I d} but the number of extra words
allocated on heap is zero.

\subsection*{\iop{i\_bs\_init\_bits\_heap I I I d}}

Allocates a heap binary of size (in bits) Arg0. If the size is not divisible by
8 than allocates a subbinary referencing the exact number of bits of the heap
binary. Makes sure that heap has Arg1 extra words available. Arg2 is Live. Puts
the result (the heap binary or the subbinary) to Arg3.

\subsection*{\iop{i\_bs\_add j I d}}

Calculates the total of the number of bits in \tmpa{} and the number of units in
\tmpb{}. Arg1 is the unit. Stores the result to Arg2.

\subsection*{\iop{i\_bs\_init\_writable}}

Allocates a shared data space of size R0. Creates a ProcBin on the heap
referencing the data. Create a subbinary referencing the ProcBin. Save the
subbinary to R0.

\subsection*{\iop{i\_bs\_append j I I I d}}

Appends \tmpa{} bits to a binary \tmpb{}. If the binary is a writable subbinary
referencing a ProcBin with enough empty space then a new writable subbinary is
created and the old one is made non-writable. In other cases, creates a new
shared data, a new ProcBin, and a new subbinary. For all heap allocation, a
space for more Arg1 words are requested. Arg2 is Live. Arg3 is unit. Saves the
resultant subbinary to Arg4.

\subsection*{\iop{i\_bs\_private\_append j I d}}

Same as \iop{i\_bs\_append j I I I d} but assumes that the binary writable.
Reallocates the shared data if there is not enough space. This should be safe.
Changes the subbinary \tmpb{} in place. No heap allocations. Arg1 is the unit.
Save the result to Arg2.

\subsection*{\iop{i\_new\_bs\_put\_integer j s I s}}

Assumes that previous instructions created a binary creation context. Packs an
integer value Arg3 into Arg1 units of the context. Arg2 are the unit and flags.
The failure label Arg0 is unused.

\subsection*{\iop{i\_new\_bs\_put\_integer\_imm j I I s}}

Assumes that previous instructions created a binary creation context. Packs an
integer value Arg3 into Arg1 bits of the context. Arg2 are flags.  The failure
label Arg0 is unused.

\subsection*{\iop{i\_bs\_utf8\_size s d}}

\begin{verbatim}
/*
  * Calculate the number of bytes needed to encode the source
  * operarand to UTF-8. If the source operand is invalid (e.g. wrong
  * type or range) we return a nonsense integer result (2 or 4). We
  * can get away with that because we KNOW that bs_put_utf16 will do
  * full error checking.
  */
\end{verbatim}

\subsection*{\iop{i\_bs\_utf16\_size s d}}

\begin{verbatim}
/*
  * Calculate the number of bytes needed to encode the source
  * operarand to UTF-16. If the source operand is invalid (e.g. wrong
  * type or range) we return a nonsense integer result (2 or 4). We
  * can get away with that because we KNOW that bs_put_utf16 will do
  * full error checking
  */
\end{verbatim}

\subsection*{\iop{i\_bs\_put\_utf8 j s}}

Assumes that previous instructions created a binary creation context. Packs a
UTF-8 character Arg1 into the context. The failure label Arg0 is unused.

\subsection*{\iop{i\_bs\_put\_utf16 j I s}}

Assumes that previous instructions created a binary creation context. Packs a
UTF-16 character Arg2 into the context. Arg1 are flags. The failure label Arg0
is unused.

\subsection*{\iop{i\_bs\_validate\_unicode j s}}

Validates a single unicode character Arg1. Raises \verb$badarg$ on error. Arg0
is not used. NB: the existence of the instruction may be dependent on the
implementation detail of binary construction; see comment in beam\_emu.c, line
3963.

\subsection*{\iop{i\_new\_bs\_put\_float j s I s}}

Assumes that previous instructions created a binary creation context. Packs a
float of Arg1 units size into the context. Arg2 are the unit and flags. The
failure label Arg0 is not used.

\subsection*{\iop{i\_new\_bs\_put\_float\_imm j I I s}}

Assumes that previous instructions created a binary creation context. Packs a
float Arg3 of Arg1 bits size into the context. Arg2 are flags. The failure label Arg0
is not used.

\subsection*{\iop{i\_new\_bs\_put\_binary j s I s}}

Assumes that previous instructions created a binary creation context. Packs a
binary Arg3 of Arg1 units size into the context. Arg2 are flags and the unit.
The failure label Arg0 is not used.

\subsection*{\iop{i\_new\_bs\_put\_binary\_imm j I s}}

Assumes that previous instructions created a binary creation context. Packs a
binary Arg2 of Arg1 bits size into the context. The failure label Arg0 is not
used.

\subsection*{\iop{i\_new\_bs\_put\_binary\_all j s I}}

Assumes that previous instructions created a binary creation context. Packs a
whole binary Arg1 into the context. Arg2 is the unit. The failure label Arg0 is
not used.

\subsection*{\iop{bs\_put\_string I I}}

Assumes that previous instructions created a binary creation context. Packs a
string of characters Arg0 bytes long referenced by the pointer Arg1 into the
context.

\subsection*{\iop{fmove qdl ld}}

Copies Arg0 to Arg1.

\subsection*{\iop{fconv d l}}

Converts Arg0 to float and places the result to Arg1.

\subsection*{\iop{i\_fadd l l l}}

$Arg2 = Arg0 + Arg1$.

\subsection*{\iop{i\_fsub l l l}}

$Arg2 = Arg0 - Arg1$.

\subsection*{\iop{i\_fmul l l l}}

$Arg2 = Arg0 \times Arg1$.

\subsection*{\iop{i\_fdiv l l l}}

$Arg2 = Arg0 / Arg1$.

\subsection*{\iop{i\_fnegate l l l}}

$Arg1 = -Arg0$.

\subsection*{\iop{i\_fcheckerror}}

Checks if the FR0 contains Nan or +-Inf. Raises \verb$badarith$ if this is the
case.

\subsection*{\iop{fclearerror}}

Maybe clears FP error flag.

\subsection*{\iop{i\_increment rxy I I d}}

$Arg3 = Arg0 + Arg1$. Arg2 is Live.

\subsection*{\iop{i\_plus j I d}}

$Arg2 = \tmpa{} + \tmpb{}$. Arg1 is Live. Jumps to Arg0 on error if Arg0 is not
zero.

\subsection*{\iop{i\_minus j I d}}

$Arg2 = \tmpa{} - \tmpb{}$. Arg1 is Live. Jumps to Arg0 on error if Arg0 is not
zero.

\subsection*{\iop{i\_times j I d}}

$Arg2 = \tmpa{} \times \tmpb{}$. Mixed. Arg1 is Live. Jumps to Arg0 on error if Arg0 is
not zero.

\subsection*{\iop{i\_m\_div j I d}}

$Arg2 = \tmpa{} / \tmpb{}$. Mixed. Arg1 is Live. Jumps to Arg0 on error if Arg0 is
not zero.

\subsection*{\iop{i\_int\_div j I d}}

$Arg2 = \tmpa{} / \tmpb{}$. Division is integer. Arg1 is Live. Jumps to Arg0 on
error if Arg0 is not zero.

\subsection*{\iop{i\_rem j I d}}

$Arg2 = \tmpa{} \% \tmpb{}$. Division is integer. Arg1 is Live. Jumps to Arg0 on
error if Arg0 is not zero.

\subsection*{\iop{i\_bsl j I d}}

$Arg2 = \tmpa{} <</>> \tmpb{}$. The direction of the shift depends on the sign of
\tmpb{}. Arg1 is Live. Jumps to Arg0 on error if Arg0 is not zero.

\subsection*{\iop{i\_bsr j I d}}

$Arg2 = \tmpa{} >>/<< \tmpb{}$. The direction of the shift depends on the sign of
\tmpb{}. Arg1 is Live. Jumps to Arg0 on error if Arg0 is not zero.

\subsection*{\iop{i\_band j I d}}

$Arg2 = \tmpa{} \& \tmpb{}$. Arg1 is Live. Jumps to Arg0 on error if Arg0 is not
zero.

\subsection*{\iop{i\_bor j I d}}

$Arg2 = \tmpa{} | \tmpb{}$. Arg1 is Live. Jumps to Arg0 on error if Arg0 is not
zero.

\subsection*{\iop{i\_bxor j I d}}

$Arg2 = \tmpa{} \oplus \tmpb{}$. Arg1 is Live. Jumps to Arg0 on error if Arg0 is not
zero.

\subsection*{\iop{i\_int\_bnot j s I d}}

$Arg3 = \neg Arg1$. Arg2 is Live. Jumps to Arg0 on error if Arg0 is not zero.

\subsection*{\iop{i\_gc\_bif1 j I s I d}}

Executes a guard BIF at the address Arg1 with a single parameter Arg2. Arg3 is
Live. Stores the result in Arg4. Jumps to Arg0 on error if it is not zero.
Otherwise, raises an exception.

\subsection*{\iop{i\_gc\_bif2 j I I d}}

Executes a guard BIF at the address Arg1 with two parameters --- \tmpa{} and
\tmpb{}. Arg2 is Live. Stores the result in Arg3. Jumps to Arg0 on error if it
is not zero.  Otherwise, raises an exception.

\subsection*{\iop{i\_gc\_bif3 j I s I d}}

Executes a guard BIF at the address Arg1 with three parameters --- Arg2,
\tmpa{}, and \tmpb{}. Arg3 is Live. Stores the result in Arg4. Jumps to Arg0 on
error if it is not zero.  Otherwise, raises an exception.

\section*{Message passing}

\begin{verbatim}
 /*
  * Skeleton for receive statement:
  *
  *             recv_mark L1                     Optional
  *             call make_ref/monitor            Optional
  *             ...
  *             recv_set L1                      Optional
  *      L1:          <-------------------+
  *                   <-----------+       |
  *                               |       |
  *             loop_rec L2 ------+---+   |
  *             ...               |   |   |
  *             remove_message    |   |   |
  *             jump L3           |   |   |
  *             ...               |   |   |
  *             loop_rec_end L1 --+   |   |
  *      L2:          <---------------+   |
  *             wait L1  -----------------+      or wait_timeout
  *             timeout
  *
  *      L3:    Code after receive...
  *
  *
  */
\end{verbatim}

\section*{Meta instructions}

All meta instructions are omitted by the loader. Their presence is recognized by
the interpreter as an error.

\subsection*{\iop{int\_code\_end}}

Maybe marks the end of the code.

\subsection*{\iop{label L}}

The label L is set to the current code location.

\subsection*{\iop{line I}}

Maybe augments debugging output when backtracking.

\section*{Superficial instructions}

The use of the following instructions supported by BEAM is avoided in Ling$:$

\begin{multicols}{2}
\begin{itemize}
\item \iop{too\_old\_compiler}
\item \iop{i\_trace\_breakpoint}
\item \iop{i\_debug\_breakpoint}
\item \iop{i\_count\_breakpoint}
\item \iop{i\_time\_breakpoint}
\item \iop{i\_return\_time\_trace}
\item \iop{i\_return\_to\_trace}
\item \iop{i\_yield}
\item \iop{normal\_exit}
\item \iop{continue\_exit}
\item \iop{apply\_bif}
\item \iop{call\_nif}
\item \iop{call\_error\_handler}
\item \iop{call\_traced\_function}
\item \iop{return\_trace}
\item \iop{unsupported\_guard\_bif}
\item \iop{on\_load}
\end{itemize}
\end{multicols}

\end{document}
